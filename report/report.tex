\documentclass{artikel3}
\usepackage[utf8]{inputenc}
\usepackage[margin=1in]{geometry}
\usepackage[titletoc,title]{appendix}
\usepackage[T1]{fontenc}
\usepackage{lmodern}
\usepackage{amsmath,amsfonts,amssymb,mathtools}
\usepackage{graphicx,float}
\usepackage{circuitikz}

\title{Transformata Fouriera}
\author{Mateusz Kojro}
\date{}

\begin{document}

\maketitle

\section{Podstawa teoretyczna}

\subsection{Transformata Fouriera}
Transformacje Fourierowskie to dziedzina transformacji pozwalających na przekształcanie
funkcji z dziedziny czasu (np.\ przebiegi natężenia dźwięku w czasie) na funkcje w dziedzinie częstotliwości (np.\ natężenia dźwięku dla poszczególnych częstotliwości). Jednowymiarową transformatę możemy zapisać jako funkcje $ f: \mathbb{R} \to \mathbb{C} $ za pomocą wzoru:

\begin{equation}
    \hat{f}(\xi) = \int_{-\infty}^{\infty} f(x) \exp{(-2 \pi i x \xi)} \ dx, \ \ \forall \xi \in \mathbb{R}
\end{equation}

gdzie $i$ oznacza jednostkę urojoną a jeżeli $x$ oznacza wartości należące do dziedziny badanej funkcji (W przykładzie badania natężenia dźwięku od czasu będzie miał jednostkę czasu), $f(x)$ jest wartością badanej funkcji dla danego $x$ a $\xi$ oznacza częstotliwość (w przypadku gdy $x$ jest czasem mierzonym w sekundach $\xi$ będzie miało jednostkę Hz)

% TODO: Poco sie to robi 

\subsection{Odwrotna transformata Fouriera}

W niektórych sytuacjach możliwe jest odwrócenie transformaty w celu uzyskania oryginalnego sygnału


\subsection{Transformaty wielowymiarowe}

\subsection{Dyskretne transformaty Fouriera}







\section{Transformaty Dyskretne}


\end{document}
