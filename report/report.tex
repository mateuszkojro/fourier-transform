\documentclass{artikel3}
\usepackage[utf8]{inputenc}
\usepackage[margin=1in]{geometry}
\usepackage[titletoc,title]{appendix}
\usepackage[T1]{fontenc}
\usepackage{lmodern}
\usepackage{amsmath,amsfonts,amssymb,mathtools}
\usepackage{graphicx,float}
\usepackage{circuitikz}

\title{Transformata Fouriera}
\author{Mateusz Kojro}
\date{}

\begin{document}

\maketitle

\section{Podstawa teoretyczna}

\subsection{Transformata Fouriera}
Transformacje Fourierowskie to dziedzina transformacji pozwalających na przekształcanie
funkcji z dziedziny czasu (np.\ przebiegi natężenia dźwięku w czasie) na funkcje w dziedzinie częstotliwości (np.\ natężenia dźwięku dla poszczególnych częstotliwości). Jednowymiarową transformatę możemy zapisać jako funkcje $ f: \mathbb{R} \to \mathbb{C} $ za pomocą wzoru:

\begin{equation}
    \hat{f}(\xi) = \int_{-\infty}^{\infty} f(x) \exp{(-2 \pi i x \xi)} \ dx, \ \ \forall \xi \in \mathbb{R}
\end{equation}

gdzie $i$ oznacza jednostkę urojoną a jeżeli $x$ oznacza wartości należące do dziedziny badanej funkcji (W przykładzie badania natężenia dźwięku od czasu będzie miał jednostkę czasu), $f(x)$ jest wartością badanej funkcji dla danego $x$ a $\xi$ oznacza częstotliwość (w przypadku gdy $x$ jest czasem mierzonym w sekundach $\xi$ będzie miało jednostkę Hz)
q
% TODO: Poco sie to robi 

\subsection{Odwrotna transformata Fouriera}

W niektórych sytuacjach możliwe jest odwrócenie transformaty w celu uzyskania oryginalnego sygnału za pomocą tzw. odwrotnej transformaty Fouriera opisanej wzorem:

\begin{equation}
f(x) = \int_{-\infty}^{\infty} \hat{f}(\xi) \exp{(2 \pi i x \xi)} \ d\xi, \ \ \forall x \in \mathbb{R}
\end{equation}

gdzie $\hat{f}$ oznacza wynik transformaty fouriera dla funkcji $f$

\subsection{Transformaty wielowymiarowe}

Transformata Fouriera może zostać uogólniona do $n$ wymiarów korzystając z wzoru:

\begin{equation}
    \hat{f}(k) = \frac{1}{{(2\pi)}^\frac{n}{2}} \int f(r) \exp{(-ikr)} d^n r
\end{equation}

w którym $k=[k_1,k_2,\ldots, k_n]$

\subsection{Dyskretne transformaty Fouriera}
Dyskretyzacja transformaty Fouriera pozwala na zastosowanie tradycyjnej transformaty do analizy sygnalów mierzonych przez instrumenty (instrument pomiarowy generować będzie dyskretne próbki danych a nie ciągłą funkcje).
Dyskretna transformatę możemy opisać za pomocą sumy przekształcającej ciąg próbek jakiegoś sygnału $[x_0, x_1, \ldots, x_{N-1}]$ gdzie $x_i \in \mathbb{R}$ w ciąg harmonicznych tego sygnału oznaczanych: $[X_0, X_1, \ldots, X_{N-1}]$ gdzie $X_n \in \mathbb{C}$ danej wzorem:

\begin{equation}
    X_k = \sum_{n=0}^{N-1} x_n \exp{\left(\frac{-i k n 2\pi}{N}\right)} , \ 0 \le k \le N - 1
\end{equation}

gdzie $k$ to numer badanej harmonicznej a $N$ to liczba próbek w sygnale.

\section{Analiza sygnału za pomocą transformaty Fouriera}

Jednowymiarowa sykretna transformata Fouriera moze zostac wykorzystana do analizy i modyfikacji funkcji przebiegu czasowego sygnalu.
Dobry przykładem takiego zastosowania jest wykorzystanie DFT podczas analizy i obróbki sygnalu dźwiękowego.
Umozliwia ona miedzy innymi na analize spektrum czestotliwosci w celu separacji sygnalow skladowych. Natomiast w polaczeniu z IDFT moze zostac wykorzytana w celu zmiany sygnalu wejsciowgo (np. w celu usuniecia szumu na danej czestotliwosci lub wzmocnienia sygnalu na innej)

% TODO: Zadeklarować gdzieś te skróty
\subsection{Analiza sprawności implementacji DFT i IDFT }

\subsubsection{Wykorzystane narzędzia}

W celu analizy sygnalu podanego w zadaniu zaimplementowane zostały DFT i IDFT. Wykorzystano język programowania C++ w standardzie 14 (ISO/IEC $14882$) kompilowany za pomocą kompilatora MSVC w wersji $19.29.30136$

\subsubsection{Badanie sygnału o znanych składowych}

W celu zbadania poprawnosci implementacji DFT wygenerowano 3 testowe sygnaly na przedziale od $0$ do $10 \pi$ kazdy z nich zawiera $1000$ probek (ich przebiegi czasowe przedstawione zostaly na rysunku) % TODO: Dodac raw sygnaly i numer rysunku 

\begin{enumerate}
    % TODO: This is not correct
    \item Funkcja określona wzorem $f(t) = \sin(t)$
    \item Funkcja określona wzorem $f(t) = \sin(2t)$
    \item Złożenie funkcji $1$ i funkcji $2$
\end{enumerate}

powinnismy wiec otrzymac maxima transformaty sygnałów w $x=5$ dla sygnalu $1$ i $x=10$ dla sygnalu $2$ o wartości około $f(x)=0.5$. Dla ich zlozenia transformata powinna natomiast wygladac jak sum tych wykresow. Wyniki przedstawione na rysunku: 
% TODO: dodac rysunek z numerem i informacje o wycieciu tylko kawalka
Z duza dokldanoscia 
% TODO: jaka

zgadzaja sie z oczekiwanymi wynikami. Co argumentuje poprawnosc implementacji dft.

W celu sprawdzenia poprawnosci IDFT wyniki tej ww. transformacji zostana nastepnie poddane transfomacji odwrotnej a uzyskany sygnal powinien byc zblizony do sygnalu oryginalnego. Wyniki IDFT porownane z sygnalem orginlanym przedstawiono na 
% TODO dodaj rysunek 

% TODO Wykres rozicy miedzy oryginalnym a odwroconym

\subsubsection{Porównanie wyników z implementacja biblioteki SciPy}

Droga zastosowana metoda badania poprawnosci implementacji jest porownanie wynikow wygenerowanych dla danego zestawu danych z wynikiami otrzymanymi po aplikowaniu implementacji dft i idft znajdujacych sie w bibilotece SciPy. 
Wykorzytsany do tego zostanie plik dane\_10.in.

% TODO Wkleic i zlinkowac orginalny
% TODO Wkleiuc i zlinkowac moj i python obok siebbie transformed
% TODO Wkleiuc i zlinkowac moj i python obok siebbie reversed
% TODO WYkres roznicy transfoemd
% TODO WYkres roznicy inversed

% TODO Jakich funckcji uzyto w pythonie

Rysunki ? jasno obrazuja bardzo wysoki stopien podobienstwa pomiedzy wynikami testowanej implementacji i implementacji w bibilotece python (wyniki fft w pythonie poddane zostaly normalizaci poprzez podzielnie wszystkich wartości przez ilosc probek poniewaz testowana implementacja stosuje taki zabieg)

% TODO: jaki stopien podobienstwa



\subsection{Analiza zadanego sygnału}

\subsubsection{Sygnal 1 wymiarowy}

\subsubsection{Sygnal 2 wymiarowy}

\subsection{Wnioski i podsumowanie}


\end{document}

% TODO: Sth about FFT