\documentclass{artikel3}
\usepackage[utf8]{inputenc}
\usepackage[margin=1in]{geometry}
\usepackage[titletoc,title]{appendix}
\usepackage[T1]{fontenc}
\usepackage{lmodern}
\usepackage{amsmath,amsfonts,amssymb,mathtools}
\usepackage{graphicx,float}
\usepackage{circuitikz}


\usepackage{csquotes}
\usepackage[style=numeric]{biblatex}
\addbibresource{references.bib}


\graphicspath{ {./imgs/} }

\title{Transformata Fouriera}
\author{Mateusz Kojro}
\date{}

\begin{document}

\maketitle

\section{Podstawa teoretyczna}

\subsection{Transformata Fouriera}
Transformacje Fourierowskie to dziedzina transformacji pozwalających na przekształcanie
funkcji z dziedziny czasu (np.\ przebiegi natężenia dźwięku w czasie) na funkcje w dziedzinie częstotliwości (np.\ natężenia dźwięku dla poszczególnych częstotliwości). Jednowymiarową transformatę możemy zapisać jako funkcje $ f: \mathbb{R} \to \mathbb{C} $ za pomocą wzoru\cite{math_phys}:

\begin{equation}
    \hat{f}(\omega) = \frac{1}{\sqrt{2 \pi}} \int_{-\infty}^{\infty} f(x) \exp{(- i \omega x)} \ dx, \ \ \forall \omega \in \mathbb{R}
\end{equation}

gdzie $i$ oznacza jednostkę urojoną a jeżeli $x$ oznacza wartości należące do dziedziny badanej funkcji (W przykładzie badania natężenia dźwięku od czasu będzie miał jednostkę czasu), $f(x)$ jest wartością badanej funkcji dla danego $x$ a $\omega$ oznacza częstotliwość (w przypadku gdy $x$ jest czasem mierzonym w sekundach $\omega$ będzie miało jednostkę Hz)
q
% TODO: Poco sie to robi 

\subsection{Odwrotna transformata Fouriera}

W niektórych sytuacjach możliwe jest odwrócenie transformaty w celu uzyskania oryginalnego sygnału za pomocą tzw. odwrotnej transformaty Fouriera opisanej wzorem\cite{math_phys}:

\begin{equation}
f(x) = \frac{1}{\sqrt{2 \pi}} \int_{-\infty}^{\infty} \hat{f}(\omega) \exp{( i \omega x)} \ d\omega, \ \ \forall x \in \mathbb{R}
\end{equation}

gdzie $\hat{f}$ oznacza wynik transformaty fouriera dla funkcji $f$

\subsection{Dyskretne transformaty Fouriera}
Dyskretyzacja transformaty Fouriera pozwala na zastosowanie tradycyjnej transformaty do analizy sygnalów mierzonych przez instrumenty (instrument pomiarowy generować będzie dyskretne próbki danych a nie ciągłą funkcje).
Dyskretna transformatę możemy opisać za pomocą sumy przekształcającej ciąg próbek jakiegoś sygnału $[x_0, x_1, \ldots, x_{N-1}]$ gdzie $x_i \in \mathbb{R}$ w ciąg harmonicznych tego sygnału oznaczanych: $[X_0, X_1, \ldots, X_{N-1}]$ gdzie $X_n \in \mathbb{C}$ danej wzorem \cite{python_numerical} :

\begin{equation}
    X_k = \sum_{n=0}^{N-1} x_n \exp{\left(\frac{-i k n 2\pi}{N}\right)} , \ 0 \le k \le N - 1
\end{equation}

gdzie $k$ to numer badanej harmonicznej a $N$ to liczba próbek w sygnale.

\subsection{Transformaty wielowymiarowe}

Transformata Fouriera może zostać uogólniona do $n$ wymiarów korzystając z wzoru \cite{math_phys}:

\begin{equation}
    \hat{f}(k) = \frac{1}{{(2\pi)}^\frac{n}{2}} \int f(r) \exp{(-ikr)} d^n r
\end{equation}

w którym $k=[k_1,k_2,\ldots, k_n]$

a jej odwrotność do 

\begin{equation}
    f(r) = \frac{1}{{(2\pi)}^\frac{n}{2}} \int \hat{f}(k) \exp{(ikr)} d^n k
\end{equation}


\subsection{Szybka transformata Fouriera} 
Tzw szybkie transformacje Fouriera to zbiór algorytmów mających na celu przspieszenie działania standardowej DFT
\cite{fft_algo_wiki}. Najczęściej stosowanym algorytmem z tej rodziny jest algorytm Cooley'a\-Tukey'a  umożliwiający poprawienie złożoności czasowej DFT z $O(n^2)$ do $O(n \log n)$\cite{marchewczyk}  wykorzystując jej wewnętrzną symetrię i używając standardowej DFT tylko dla próbek o długości mniejszej od pewnego zadanego maksimum. Jego ograniczeniem jest wymaganie aby długość sygnału była potęgą 2\cite{python_numerical}.  


% TODO: FFT section troche o 


\section{Analiza sygnału za pomocą transformaty Fouriera}

Jednowymiarowa sykretna transformata Fouriera może zostać wykorzystana do analizy i modyfikacji funkcji przebiegu czasowego sygnalu.
Dobry przykładem takiego zastosowania jest wykorzystanie DFT podczas analizy i obróbki sygnalu dźwiękowego.
Umożliwia ona miedzy innymi na analizę spektrum częstotliwości w celu separacji sygnałów składowych. Natomiast w połączeniu z IDFT moze zostać wykorzystana w celu zmiany sygnalu wejściowego (np. w celu usuniecia szumu na danej częstotliwości lub wzmocnienia sygnalu na innej)

% TODO: Zadeklarować gdzieś te skróty
\subsection{Analiza sprawności implementacji DFT i IDFT }

\subsubsection{Wykorzystane narzędzia}

W celu analizy sygnalu podanego w zadaniu zaimplementowane zostały DFT i IDFT. Wykorzystano język programowania C++ w standardzie 14 (ISO/IEC $14882$) kompilowany za pomocą kompilatora MSVC w wersji $19.29.30136$

\subsubsection{Badanie sygnału o znanych składowych}

W celu zbadania poprawności implementacji DFT wygenerowano 3 testowe sygnały na przedziale od $0$ do $10 \pi$ każdy z nich zawiera $100$ probek (ich przebiegi czasowe przedstawione zostały na rysunku) 
% TODO: Dodać raw sygnały i numer rysunku 

\begin{enumerate}
    % TODO: This is not correct
    \item Funkcja określona wzorem $f(t) = \sin(t)$
    \item Funkcja określona wzorem $f(t) = \sin(2t)$
    \item Złożenie funkcji $1$ i funkcji $2$
\end{enumerate}


\begin{figure}[H]
    \centering
    \includegraphics[width=1.00\textwidth]{wave_raw.png}
\end{figure}

powinniśmy wiec otrzymać maxima transformaty sygnałów w $x=5$ dla sygnalu $1$ i $x=10$ dla sygnalu $2$ o wartości około $f(x)=0.5$. Dla ich złożenia transformata powinna natomiast wyglądać jak sum tych wykresów. Wyniki przedstawione na rysunku ? 
Z duża dokładnością zgadzaja sie z oczekiwanymi wynikami. Co argumentuje poprawność implementacji dft.

\begin{figure}[H]
    \centering
    \includegraphics[width=1.00\textwidth]{wave_transformed.png}
\end{figure}


% TODO: dodać rysunek z numerem i informacje o wycięciu tylko kawalka
% TODO: jaka dokladnoscia

W celu sprawdzenia poprawności IDFT wyniki tej ww. transformacji zostaną nastepnie poddane transformacji odwrotnej a uzyskany sygnał powinien byc zbliżony do sygnalu oryginalnego. Wyniki IDFT porównane z sygnalem oryginalnym przedstawiono na 

\begin{figure}[H]
    \centering
    \includegraphics[width=1.00\textwidth]{wave_reversed.png}
\end{figure}

% TODO dodaj rysunek 

% TODO Wykres ronicy miedzy oryginalnym a odwróconym

\subsubsection{Porównanie wyników z implementacja biblioteki SciPy}

Droga zastosowana metoda badania poprawności implementacji jest porównanie wyników wygenerowanych dla danego zestawu danych z wynikiami otrzymanymi po aplikowaniu implementacji dft i idft znajdujących sie w bibilotece SciPy \cite{scipy_fft}. 
Wykorzystany do tego zostanie plik dane\_10.in.

\begin{figure}[H]
\centering
\includegraphics[width=1.00\textwidth]{raw_python_comp.png}
\end{figure}


% TODO Wkleić i linkować oryginalny
% TODO Wkleić i linkować moj i python obok siebie transformed
% TODO Wkleić i linkować moj i python obok siebie reversed
% TODO WYkres rożnicy transformed
% TODO WYkres rożnicy inverses

% TODO Jakich funkcji użyto w pythonie

Rysunki ? jasno obrazują bardzo wysoki stopień podobienstwa pomiędzy wynikami testowanej implementacji i implementacji w bibilotece python (wyniki fft w pythonie poddane zostaly normalizaci poprzez podzielnie wszystkich wartości przez ilosc probek poniewaz testowana implementacja stosuje taki zabieg)

\begin{figure}[H]
    \centering
    \includegraphics[width=1.00\textwidth]{transfomed_pcomp.png}
\end{figure}
    


\begin{figure}[H]
    \centering
    \includegraphics[width=1.00\textwidth]{inversed_pcomp.png}
\end{figure}
    

% TODO: jaki stopien podobienstwa

\subsection{Analiza zadanego sygnału}

% TODO Dodac jego raw rysunki
Na rysunkach ? przedstawiono przebiegi czasowe zadanych sygnalow ($10$, $33$ i dwu wymiarowy $6$) ktore nastepnie poddane zostana analizie.

\begin{figure}[H]
    \centering
    \includegraphics[width=1.00\textwidth]{10_raw.png}
\end{figure}

\begin{figure}[H]
    \centering
    \includegraphics[width=1.00\textwidth]{10_transformed.png}
\end{figure}


\begin{figure}[H]
    \centering
    \includegraphics[width=1.00\textwidth]{33_raw.png}
\end{figure}
\begin{figure}[H]
    \centering
    \includegraphics[width=1.00\textwidth]{33_transformed.png}
\end{figure}



\subsubsection{Sygnal 1 wymiarowy}

\subsubsection{Sygnal 2 wymiarowy}

\subsection{Wnioski i podsumowanie}

\pagebreak

\printbibliography

\end{document}

% TODO: Sth about FFT